% Configuración a 2 columnas
\documentclass[3p,twocolumn]{elsarticle}
% Paquete con el tipo de codificación del lenguaje
\usepackage[utf8]{inputenc}
% Paquete para mostrar el número de líneas
\usepackage[modulo]{lineno}
\linenumbers
% Paquete para resaltar enlaces y referencias
\usepackage{hyperref}
% Configuración del paquete
\hypersetup{
  breaklinks=true,  
  colorlinks=true,
  urlcolor=urlcolor,
  linkcolor=linkcolor,
  citecolor=citecolor,
}
% Modifica el pie de la primera página para poner el número de la misma
\makeatletter
\def\ps@pprintTitle{%
   \let\@oddhead\@empty
   \let\@evenhead\@empty
   \def\@oddfoot{\reset@font\hfil\thepage\hfil}
   \let\@evenfoot\@oddfoot
}
\makeatother
% Paquete para cambiar color texto
\usepackage[dvipsnames]{xcolor}


% Inicio del documento
\begin{document}

\title{TFG} % Título del TFG

\author[1]{Aarón Casado Monge} % Autor del TFG
\ead{aaron.casado@uah.es} % Correo del alumno

\author[2]{Juan José Cuadrado Gallego} % Tutor del TFG
\ead{jcg@uah.es} % Correo del profesor

% Dirección de la Escuela Politécnica Superior de UAH
\address{University of Alcala, Polytechnic School, Computer Science Department, Scientific and Technological Campus, Politechnic Building. Office: O243, 28805, Alcala de Henares, Madrid, Spain}

% Abstracto del TFG
\begin{abstract}
Clusterización (qué es y para qué sirve) - Técnicas (para qué sirven) - Paquetes 
\end{abstract}

% Palabras clave del TFG
\begin{keyword}
BigData, Statistics, Bioestatistics, Data Science, Clustering
\end{keyword}

\maketitle % Creación del Título, autores, abstracto...
 
% Introducción del TFG
\section{Introducción}

% Era de la información, gran volumen de datos, necesidad de analizarlos, aparición Big Data, Data Science y Data mining. Clustering como método especial para analizar los datos y agruparlos sin saber la respuesta.

Desde finales del sigo \textsc{xx} se ha considerado que vivimos en la ``era de la información", una etapa caracterizada por el incremento, desarrollo y propagación de emergenetes tecnologías de la información y comunicación que han permitido al ser humano romper las barreras de la distancia, el tiempo y el lugar a la hora de comunicarse y compartir información; actividades que han sido decisivas en nuestra historia \cite{cita1}. Sin embargo, la era en la que realmente vivimos es la ``era de los datos"; cada día se generan más de dos mil quinientos millones de petabytes \footnote {Un Petabyte es una unidad de información o almacienamiento de datos equivalente a un cuadrillon de bytes, mil terabytes o un millón de gigabytes. En este caso, es el equivalente a 2.5 quintillones de bytes.} de datos provenientes de negocios, ciencias, Internet y casi cualquier actividad del día a día \cite{cita2} que acaban volcados en redes de ordenadores, sitios web, bases de datos y otros medios de almacenaje. 

Esta gran cantidad de datos proviene de lo computerizada que está la sociedad. Continuar con el libro.


%\textbf{Clustering} es uno de los diversos métodos que ofrece \textbf{Data Mining}, una rama de \textbf{Data Science} (Ciencia de los Datos), que se ha posicionado rápidamente como una de las disciplinas más influyentes en plena era de la información, donde el volumen de datos generado diariamente y que se almacenan en bases de datos es inmenso, pero para poder darle uso, es necesario organizarlos y agruparlos de manera adecuada (Data Warehousing), analizarlos para obtener información y conocimiento de los mismos (Data Mining) y presentar los resultados de manera apropiada (Visualización). 

%Aunque Data Science es todavía joven y carece de una definición oficial, podríamos definirla como la unión de la Estadística, la Inteligencia Artificial y la Programación aplicados a ese grán volumen de datos (Big Data \footnote{Definir Big Data}) permitiendo extraer conocimiento a partir de ellos \cite{cita4}.

\clearpage
\section{Clustering} 

% Origen de la clusterización: Cluster analysis was originated in anthropology by Driver and Kroeber in 1932[1] and introduced to psychology by Joseph Zubin in 1938[2] and Robert Tryon in 1939[3] and famously used by Cattell beginning in 1943[4] for trait theory classification in personality psychology.


Clustering o Cluster Analysis, adaptado al español como \textbf{Clusterización}, Agrupamiento o Análisis de Grupos es un método de clasificación no supervisada perteneciente a Data Mining, que busca definir, para una característica determinada o Suceso Elemental (SE) \footnote{Definir suceso elemental}, un conjunto de grupos de observaciones (suceso) \footnote{Definir Suceso} con valores cercanos. Estos grupos son los denominados clusters o grupos y permiten a partir de los diferentes sucesos elementales que configuran dicho suceso, asignar dicho SE al mismo. Clustering nos permite definir los valores de cada cluster durante el proceso de clasificación \cite{cita5}.


\newpage

\section{Referencias}
\renewcommand{\section}[2]{}
\begin{thebibliography}{X}

\bibitem{cita1} Alberts, D. S., \& Papp, D. S. (1997). \href{http://www.dodccrp.org/files/Alberts_Anthology_I.pdf} {The information age: An anthology on its impact and consequences}. Office of the Assistant Secretary of Defense Washington DC Command and Control Research Program (CCRP).

\bibitem{cita2} Becoming A Data-Driven CEO | Domo. (2018). Data never sleeps 6.0 \href{https://www.domo.com/solution/data-never-sleeps-6} {https://www.domo.com/solution/data-never-sleeps-6}

\bibitem{cita3} Everyman's Science

\bibitem{cita4} Apuntes JJ Data Science

\bibitem{cita5} Apuntes JJ clustering



Han, J., Kamber, M., \& Pei, J. (2012). Data Mining: Concepts and
Techniques (3rd ed., p.~740). 225 Wyman Street, Waltham, MA 02451, USA:
Morgan Kaufmann Publishers, Elsevier.

\end{thebibliography}

\section{Ayuda}

https://normas-apa.org/referencias/citar-diccionario/

https://www.scribbr.es/detector-de-plagio/

https://tablesgenerator.com/

elsevier dos páginas latex

\end{document}